\subsection{Obiettivi, servizi, requisiti utente, requisiti del Sistema, definizioni dei requisiti, specifiche, definizione degli obiettivi, definizione dei requisiti, specificazione dei requisiti, document dei requisiti, dominio applicativo, stakeholder, analista sw, progettista sw, prospettiva dominio applicativo e tecnica}
\subsection{Requisiti funzionali e Non funzionali; domain requirement. Imprecisione/ambiguità dei requisiti; completezza, consistenza, correttezza, Regola delle 3 C. Requisiti non funzionali di prodotto, di processo ed esterni}
\subsection{Obiettivi e requisiti, verificabilità dei requisiti. Metriche per misurare i requisiti. Interazione e conflitti nei requisiti. Req. di dominio: non facilmente comprensibili e impliciti}
\subsection{Linguaggi di rappresentazione di Requisiti e Specifiche, con relative problemi: linguaggio naturale (chiarezza, confuzione, amalgamazione di più requisiti); linguaggio strutturato, suggerimenti per l’uso del linguaggio}
\subsection{Relazioni Requisiti / progetti: prima definizione dell’architettura sw}
\subsection{Problemi del LN per la specificazione: ambiguità, estrema flessibilità, linearità e mancanza di meccanismi di modularizzazione}
\subsection{Modi per strutturare il linguaggio: scrittura standardizzata, template, form, tabelle, modellli grafici, Use Case UML, sequence diagram, PDL (pseudolinguaggi)}
\subsection{Specificazione in PDL delle interface}
\subsection{Documento dei Requisiti (DdR), destinatari del document, caratteristiche (requisiti) del DdR, standard, struttura}