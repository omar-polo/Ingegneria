\subsection{Reti di Petri (RdP) posti-transizioni. Posti, transizioni, funzione di input, funzione di output, terminologia e sintassi, token, marcatura e stato corrente. Rappresntazione distribuita dello stato del sistema, evoluzione rappresntata mediante transizioni che hanno effetto sulle varie parti dello stato (posti). Regole di evoluzione. Condizioni per lo sctto di una transizione. Sincronizzazione di processi concorrenti, sincronizzazione con eventi esterni. Aspetti da non considerare in una RdP: trasformazioni I/O, principi di conservazione. Strutture di controllo rappresntate mediante specifiche semplici configurazioni di RdP}