\subsection{A cosa serve modellare, quali le diverse prospettive di modellizzazione, processi mentali di analisi e di sintesi}
\subsection{Metodologie di analisi e progettazione SW strutturate, carateristiche e criticità
Tipi di modelli}
\subsection{Modelli del contesto: diagrammi del contesto e modelli dei processi del contesto}
\subsection{Modelli comportamentali che rappresentano l’elaborazione dei dati. DFD, dati processi, flussi di dati, archivi, entità esterne. Terminologia dei dati e dei processi. Regole per la scomposizione in processi in processi di maggior dettaglio}
\subsection{Modelli comportamentali che rappresentano l’evoluzione mediante stati. Automi a stati finiti, limitazioni. Reti di Petri. Esempi ed esercizi}
\subsection{Modelli per la rappresentazione della semantica dei dati. UML con associazioni (e commenti).E-R models (vedi corso DB). Data dictionary}
\subsection{Object model: varie tipologie di diagrammi (vedi lez. Dr. Baruzzo). Ragionamento classificatorio, eredità}
