\subsection{Processi sw primari, di supporto e di gestione; loro relazioni}
\begin{description}
    \item[Processi primari] attività necessarie a specificare, progettare, produrre, mantenere ed estendere il prodotto SW
    \item[Processi di supporto] vengono eseguiti parallelamente ai processi primari al fine di supportarli e garantire la qualità ed il successo del progetto 
    \item[Processi di gestione] attività necessarie a gestire il progetto di sviluppo del prodotto SW
\end{description}
\subsection{Ciclo di Vita, definizione, caratteristiche, organizzazione}
\begin{itemize}
    \item Il Ciclo di Vita specifica come sono definiti e organizzati i vari task coinvolti nella progettazione, costruzione e manutenzione di un prodotto (artifact). Più precisamente è uno schema di riferimento che specifica in modo astratto e generale quali task svolgere e quando svolgerli.
    \item Il livello di definizione del ciclo di vita si presenta:
        \begin{itemize}
            \item astratto, poiché prescinde da dettagli tecnico/implementativi
            \item generale, poiché solitamente ricomprende una vasta gamma di casistiche ed è indipendente da contesti specifici
        \end{itemize}
    \item Organizzazione: il ciclo di vita è basato su una struttura a due livelli: il livello delle fasi con i passi principali e il livello dei task con i compiti da svolgere per ogni fase. Le fasi sono raggruppate in macrofasi e i task sono raggruppati in step
\end{itemize}
\subsection{Metodologia di sviluppo, definizione e caratteristiche}
\begin{description}
    \item[Metodologia di sviluppo] è un insieme integrato di metodi con i quali eseguire effettivamente tutti i singoli task del ciclo di vita.
    \item[Livello di definizione] di una metodologia di sviluppo si presenta:
    \begin{description}
        \item[Concreto], poiché riguarda tutti i dettagli tecnico/implementativi
        \item[Specifico] poiché solitamente tiene conto dello specifico contesto in cui è definita la metodologia
    \end{description}
\end{description}
\subsection{Metodi tecnici e metodi di gestione}
Un metodo è un insieme integrato di:
\begin{itemize}
    \item tecniche
    \item procedure
    \item linguaggi, notazioni
    \item strumenti
    \item standard, formati
    \item documentazione
    \item best practice
    \item criteri, linee guida, vincoli
\end{itemize}