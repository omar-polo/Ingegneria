\subsection{Attività generiche presenti nel processo di RE: elicitazione, analisi, specificazione, validazione, gestione; diverse prospettive nella rappresentazione del RE}
\subsection{Studio di Fattibilità: obiettivi, modalità di realizzazione e risultati}
\subsection{Elicitazione e analisi: partecipanti e ruoli, stakeholder; criticità; processo iterative di elicitazione e analisi, rappresentazione a spirale; attività del processo: discovery, classificazione e organizzazione, prioritizzazione e negoziazione, documentazione}
\subsection{R. discovery: gathering+distilling; sorgenti; risutati: reqs.di dominio, altri reqs., system models ottenuti mediante diverse prospettive di analisi (partizionamento, classificazione e proiezione)}
\subsection{Elicitazione basata sull’analisi di diversi punti di vista; tipologie di punti di vista (interazione, indiretto, di dominio); modalità di identificazione dei punti di vista: raccolta, classificazione, identificazione, analisi, documentazione; gerarchie di punti di vista}
\subsection{Tecniche di elicitazione. Interviste. Storie. Scenario (event scenario). Use case. Sequence Digram. Etnografia (osservazione dei processi). Criticità dei fattori organizzativi/politici/sociali}
\subsection{Validazione dei Requisiti. Parametri da controllare per validare. Tecniche di validazione: review, prototipizzazione, generazione di casi di test dei requisiti, metodi automatici}
\subsection{Misurazione dei requisiti: modalità e profili di misura}
\subsection{Gestione dei requisiti. Motivazione del cambiamento. Categorie di requisiti dal punto di vista del cambiamento: duraturi e volatili. Varie tipologie di r. volatili. Pianificazione della gestione. Tracciabilità}