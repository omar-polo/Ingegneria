\subsection{Cos’è un sistema distribuito (S.D.). Caratteristiche/Obiettivi, vantaggi, criticità. Aspetti progettuali da considerare. 2 tipologie: sistemi client-server e sistemi ad oggetti distribuiti. Middleware}
\subsection{Sistemi Multiprocessore. Pattern master-slave. Architettura logica e architettura fisica. Tipologie di client-server, architettura a 3 strati di riferimento, caratteristiche, vantaggi e limitazioni: two-tier, thin e fat client, distinzione ed evoluzione; three-tier; multi-tier. Distributed object architecture. Ruolo del middleware, CORBA}
\subsection{Inter-organizational computing. Sistemi peer-to-peer, organizzazione decentralizzate e semicentralizzate. Saas: caratteristiche, struttura generale, configurabilità, problematiche della multi-tenancy. SOA, reusable Web service, scenario generale, vantaggi, Standard XML-based dei Web Service, Service-oriented sw engineering.: idea e esempi. Linee guida per l’integrazione e l’interoperabilità e relativi strumenti}
\subsection{Applicazione delle architetture client-server e delle architetture SOA ai Legacy Systems: scrapper, Web service wrapper e approccio 3-tier}
\subsection{Service oriented software engineering, sviluppatori di servizi (e Web Service) e sviluppatori di applicazioni basate su servizi (Web service)}